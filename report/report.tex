\documentclass{article}
    \usepackage{amsthm, amsmath}
    \usepackage{xeCJK}
    \usepackage{algorithm}
    \usepackage{algorithmicx}
    \usepackage{algpseudocode}
    \usepackage{hyperref}
    \usepackage{listings}
    
    \title{编译原理: PA5}
    \author{娄晨耀, 2016011343}
    \date{}
    
    
    \theoremstyle{plain}
    \newtheorem{thm}{Theorem}
    
    \theoremstyle{definition}
    \newtheorem{alg}{Algorithm}
    
    
    \begin{document}
    \maketitle

    \section{实现}

    首先在GraphColorRegisterAllocator 的 alloc 开始,要恢复变量。需要调用load吧liveUse的加载进来。注意设置新加载进来的语句的liveOut。

    对于所有定值的寄存器,也就是大部分语句的op0,连边op0 到该语句的liveOut 就可以了。因为这里的op0 一定要分配一个寄存器,但是又不能和要向后传递的存器同时存在。
    
    注意处理自环可以跳过。因为自环可能出现的场景比如T4 = T4 + 1,我们并不需要改变T4 分配的寄存器。同时可以判一下重边,然后给跳过。
    
    不过这里我们每个基本块都会spill 全部变量出去,所以如果关心一下哪些liveOut 是这个块内需要的变量,可以提高一些速度,但是不本质。

    \section{spill}

    首先我们可以给Temp 增加一个叫做spillOut 的变量,表示这个变量需要spill 到栈上。同时我们Mips.java 设置空闲寄存器的时候,再多留一个寄存器,用来恢复spill 变量临时存放。
    
    可以在首先在GraphColorRegisterAllocator里面findReg 的时候,发现是spill 的变量,就插入一条load 语句和store 语句到该语句的前后即可。

    至于选择那些变量spill出去,我们可以贪心的,去统计每个变量的读写次数,然后不断选择读写最不频繁的,设置成spillOut,再把这个点删掉重新尝试染色。

    \end{document}